\problemname{Bounding Robots}

% Simulate a robot walking among obstacles which might block its path.
%  Determine where it ends up after it finishes walking, and where it thinks it
%  ends up (as if the obstacles didn't exist).

As you design your new household helper robot, the SERVE-O-MATIC 1000, you are
running it through a number of tests. In one test, you want
to make sure that it always knows where it is. In particular, you're concerned
what will happen if it's not aware of a wall, which it then runs into.
For example, if it tries to walk forward 10 meters and runs into a wall it
didn't know about after 3 meters, then there will be a 7-meter difference in
where it stopped and where it {\em thinks} it stopped. Your robot does not yet
have the intelligence to incorporate new information on the fly, such as this
new wall, so it still thinks it walked 10 meters.

To test how bad this problem might be, you test a virtual robot in a
virtual room, where the robot is not aware of the size of the room. The robot
chooses a walking course, and your program simulates the robot's walk,
keeping track of where it thinks it is, and where it actually is (by preventing
it from walking through walls).  After the simulation finishes, you want to know
how far off the robot is from where it thinks it is.

\section*{Input}

Input consists of up to 100 simulations. 

Each simulation starts with a room description, which is two 
integers $w$ and $l$ (for width and length of the rectangular room, in meters).
Each is in the range $[2, 100]$.
The robot may walk anywhere in the rectangle marked by the corners (0, 0) to
($w-1$, $l-1$), including the edges (so it may walk to (0,0) or
($w-1$, 0), for example). The robot always starts at (0, 0), and its
steps are exactly one meter long.

After the room description is a description of the path the robot plans to
take. This description starts with $1 \le n \le 100$, the number of walks the
robot plans to make. This is followed by $n$ walk descriptions to be followed
in order. Each is given on a line as {\tt x y}, where {\tt x} is one of {\tt
u}, {\tt d}, {\tt l}, or {\tt r} for up, down, left, or right, respectively,
and {\tt y} is number of meters moved in that direction (in the range $[0,
30]$).  Up and down move along the length of the room, left and right along the
width.  Up and right are in the positive direction, down and left are in the
negative direction.

Input ends when $w$ and $l$ are zero.

\section*{Output}

For each simulation, output where the robot thinks it is as 
{\tt Robot thinks x y}, and where it actually is as {\tt Actually at x y} (for
appropriate coordinate values x and y). Output a blank line after each
simulation.
