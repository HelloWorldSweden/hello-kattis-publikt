\problemname{Apaxiaaaaaaaaaaaans!}

The ancient and mysterious Apaxian civilization, which we most certainly did not make up, continues to confound the researchers at the Oriental Institute. It turns out that the Apaxians had a peculiar naming system: the more letters in your name, the higher your status in society. So, in Apaxian society, \texttt{bob} was probably a lowly servant, and \texttt{bobapalaxiamethostenes} was likely a High Priest or Minister. Even more than that, Apaxians valued the number of adjacent letters that were the same in a name. So,  while \texttt{bob} continues to be an unimpressive name, \texttt{boooooooooob} probably elicited cheers and applause wherever he went.

Unfortunately, this makes the task of reading Apaxian scrolls very cumbersome, specially when you consider that a particularly famous Apaxian queen had ten thousand consecutive \texttt{a}'s in her name. Legend has it that she was already two years old by the time the Royal Herald finished announcing her birth.

To make the Oriental Institute's life easier, the Department of Computer Science has offered to convert the Apaxian scrolls into a more readable format. Specifically, we will be taking Apaxian names and replacing all consecutive runs of the same letter by a single instance of such letter.

So, for example, the compact version of \texttt{boooob} would be \texttt{bob}, where the four consecutive \texttt{o}'s have been replaced with a single \texttt{o}. Similarly, the compact version of \texttt{bbbooobbb} would also be \texttt{bob}. On the other hand, the compact version of \texttt{bob} is still \texttt{bob}.

\section*{Input}

The input contains a single name. Each name contains only lowercase letters (\texttt{a}--\texttt{z}), no whitespace, a minimum length of 1 character, and a maximum length of 250 characters. 

\section*{Output}

The output contains the compact version of the name: any time the same letter appears two or more times in sequence, it must be replaced by a single instance of that letter.
