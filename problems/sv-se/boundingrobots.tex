\problemname{Bounding Robots}

% Simulate a robot walking among obstacles which might block its path.
%  Determine where it ends up after it finishes walking, and where it thinks it
%  ends up (as if the obstacles didn't exist).

När du designar din nya hushållshjälpsrobot SERVE-O-MATIC 1000 kommer du testa
den med ett antal olika prov. I ett prov vill du vara säker på den alltid
vet var den befinner sig. Du är speciellt angelägen om vad som händer när
den inte är medveten om en vägg, som den då går in i. Till exempel: om den
försöker gå framåt tio meter och går in i en väg den inte visste var där efter 
tre meter, då kommer det vara en sju meter skillnad mellan var den stannade
och var den {\em tror} att den stannade. Din robot är ännu inte smart nog
att ta in ny information medan den går, som då denna vägg, så den tror att 
den har gått tio meter.

För att pröva hur illa detta problem är testar du en virtuell robot i ett 
virtuellt rum, där roboten inte vet rummets storlek. Roboten väljer en 
väg att gå, och ditt program simulerar robotens gång och håller koll på var
den tror att den befinner sig, och var det faktiskt är (genom att stoppa den
från att gå igenom väggar). Efter att simuleringen avslutas vill du veta hur
långt bort roboten är från var den tror att den är.

\section*{Indata}

Indata består utav 100 simuleringar.

Varje simulering börjar med en beskrivning av rummet, vilket är två heltal
$w$ och $l$ (för bredden och längden av det rektangulära rummet, i meter).
Båda heltal ligger i intervallet $[2, 100]$.
Roboten kan gå varsomhelst i rektangeln utmärkt av hörnen (0, 0) till
($w-1$, $l-1$), inklusive kanterna (så den kan gå till (0, 0) eller 
($w-1$, 0), till exempel). Roboten startar alltid i (0, 0) och dess steg
är precis en meter långa.

Efter beskrivningen av rummet kommer en beskrivning av vägen roboten
planerat att gå. Denna beskrivning börjar med ett tal $1 \le n \le 100$,
antalet steg roboten planerar att ta. Detta följs av $n$ stegbeskrivningar som
ska göras i följd. Varje stegbeskrivning ger en linje {\tt x y},
där {\tt x} är något utav {\tt u}, {\tt d}, {\tt l}, eller {\tt r} för 
respektive riktning upp, ner, vänster eller höger. Talet {\tt y} är antalet
meter roboten rör sig i den riktningen (i intervallet $[0, 30]$). Upp och
ner är längs med rummets längd, och vänster och höger är längs med dess
bredd. Upp och höger är i positiv riktning, medan ner och vänster är i
negativ riktning.

Indata avslutas när $w$ och $l$ är noll.

\section*{Utdata}

För varje simulering är utdata var roboten tror att den är, som i 
{\tt roboten tror x y}, och var den faktiskt är, återigen som i
{\tt roboten är i x y} (där x och y är utbytta med passande värden.
Skriv ut en blankrad efter varje simulering.
