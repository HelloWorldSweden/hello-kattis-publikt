\problemname{ACM Contest Scoring}

Vårt nya svarssystem till tävlingen håller en kronologisk logg
över alla svar inlagda av varje lag under tävlingen. För varje
svar registrerar systemet antalet minuter av tävlingen som gått vid
inlämning, den bokstav som identifierar tävlingsproblemet, och
resultatet av att rätta svaret (vilket för enkelhets skull betecknas
antingen {\tt rätt} eller {\tt fel}).

Ett exempel är följande hypotetiska logg vid en specifik tid:

\vspace{-12pt}
\begin{verbatim}
3 E rätt
10 A fel
30 C fel
50 B fel
100 A fel
200 A rätt
250 C fel
300 D rätt
\end{verbatim}
\vspace{-12pt}

Rankningen av lagen relativt varandra bestäms genom en primär och 
en sekundär poängskala, beräknad från svarsdatan. Den primära skalan
är antal problem korrekt lösta. Den sekundära skalan är baserad på en kombination
av tidsåtgång och straff. Specifikt så är ett lags tidspoäng lika med
summan av inlämningstider som resulterade i {\tt rätt} svar, plus
ett 20-minuters straff för varje inlämnat {\tt fel} svar av ett problem som
i slutänden besvaras korrekt. Om inget svar löses korrekt är tidspoängen
~$0$.

I exemplet ovan ser vi att laget har korrekt besvarat tre problem:
{\tt E} på deras första försök ($3$~minuter in i tävlingen);
{\tt A} på deras tredje försök på problemet ($200$~minuter in i
tävlingen); och {\tt D} på deras första försök på problemet
($300$~minuter in i tävlingen).

Detta lags tidspoäng (inklusive straff) är $543$. Detta är
summan av: $3$~minuter för att de löste {\tt E}; $200$~minuter för
att de löste {\tt A}, med ett tillägg på $40$~straffminuter för
två tidigare felaktiga inlämningar på problemet; och tillsist
$300$~minuter för att de löste {\tt D}.
%
Lägg märke till att laget också lämnade in svar på {\tt B} och
{\tt C}, men lyckades aldrig lösa dem, och fick därför inga
straff för dessa försök.

Enligt tävlingsreglerna, efter att ha löst ett specifikt problem
ignoreras alla andra följande svar på samma problem (och kommer
därför inte med i loggen). Eftersom tid är diskretiserat i hela minuter,
kan mer än ett svar lämnas in på samma problem under en minut, men
de sparas kronologiskt, så att endast det sista svaret kan vara korrekt.

Som ett andra exempel, studera följand svarslogg:%

\vspace{-12pt}
\begin{verbatim}
7 H rätt
15 B fel
30 E fel
35 E rätt
80 B fel
80 B rätt
100 D fel
100 C fel
300 C rätt
300 D fel
\end{verbatim}
\vspace{-12pt}

Detta lag löste fyra problem, och deras totala tidspoäng (inklusive staff)
är $502$, med $7$~minuter för {\tt H}, $35+20$ för {\tt E},
$80+40$ för {\tt B}, och $300+20$ för {\tt C}.

\section*{Input}

Input innehåller $n$ rader där $0 \leq n \leq 100$, där varje rad
beskriver ett specifikt svarsinlägg. Ett svarsinlägg har tre delar:
ett heltal $m$, där $1 \leq m \leq 300$, som betecknar antal gångna minuter
då svaret togs emot; en versal (stor bokstav) som betecknar det
specifikaproblemet; och ett ord som är antingen {\tt rätt} eller {\tt fel}.
Heltalen kommer vara i icke-minskande ordning och kan innehålla
repetitioner. Efter alla logginlägg är en rad som innehåller endast
talet $-1$.

\section*{Output}

Svara med två heltal på en rad: antal lösta problem och den totala
tidspoängen (inklusive straff).
