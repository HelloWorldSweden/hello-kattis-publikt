\problemname{A New Alphabet}

\illustration{0.3}{typewriter}{Photo by \href{https://www.flickr.com/photos/zionfiction/9588998186}{r. nial bradshaw}}

Ett Nytt Alfabet har utvecklats för kommunikation över Internet. Även om dessa hieroglyfer inte nödvändigtvis 
förbättrar vår kommunikation, får de oss att känna oss \emph{coolare}.

Du har fått i uppgift att skapa ett översättningsprogram för att snabba upp bytet till vårt mer \emph{elite}
Nytt Alfabet genom att automatiskt översätta tecken skrivna i ASCII plaintext till vår nya uppsättning tecken.

Detta nya alfabet översätter en-till-flera (alltså: vissa tecken i det vanliga
svenska alfabetet kan översättas till $1$ tecken, men andra kan också översättas till
upp till $6$ olika tecken): Se tabellen nedan:

\begin{center}
    \begin{tabular}{|c|c|l||c|c|l|}
        \hline
        Original & Ny & Beskrivning & Original & Ny & Beskrivning\\
        \hline
        \verb+a+ & \verb+@+         & snabel-a                                &  \verb+n+ & \verb+[]\[]+     & hakparenteser, omvänt snedstreck, hakparenteser\\
        \verb+b+ & \verb+8+         & siffran åtta                               &  \verb+o+ & \verb+0+         & siffran noll \\
        \verb+c+ & \verb+(+         & vänsterparentes &  \verb+p+ & \verb+|D+        & vertikalstreck, stora D\\
        \verb+d+ & \verb+|)+        & vertikalstreck, högerparentes                    &  \verb+q+ & \verb+(,)+       & parenteser, kommatecken, parenteser \\
        \verb+e+ & \verb+3+         & siffran tre &  \verb+r+ & \verb+|Z+        & vertikalstreck, stora Z \\
        \verb+f+ & \verb+#+         & nummertecken (hashtagg)                        &  \verb+s+ & \verb+$+         & dollar-tecken\\
        \verb+g+ & \verb+6+         & siffran sex                                 &  \verb+t+ & \verb+']['+      & citattecken, hakparenteser, citattecken \\
        \verb+h+ & \verb+[-]+       & hakparentes, bindestreck, hakparentes &  \verb+u+ & \verb+|_|+       & vertikalstreck, understreck, vertikalstreck\\
        \verb+i+ & \verb+|+         & vertikalstreck                                       &  \verb+v+ & \verb+\/+        & omvänt snedstreck, snedstreck \\
        \verb+j+ & \verb+_|+        & understreck, vertikalstreck                           &  \verb+w+ & \verb+\/\/+      & fyra (varierande) snedstreck \\
        \verb+k+ & \verb+|<+        & vertikalstreck, mindre än-tecken &  \verb+x+ & \verb+}{+        & klammerparenteser \\
        \verb+l+ & \verb+1+         & siffran ett &  \verb+y+ & \verb+`/+        & grav accent, snedstreck\\
        \verb+m+ & \verb+[]\/[]+    & hakparenteser, snedstreck, hakparenteser               &  \verb+z+ & \verb+2+         & siffran två\\
        \hline
    \end{tabular}
\end{center}
Till exempel, att översätta textsträngen ``Hello World!'' skulle resultera i:
\begin{center}
\verb+[-]3110 \/\/0|Z1|)!+
\end{center}
Notera att gemener och versaler alla konverteras, och att alla andra tecken är oförändrade (i detta exempel alltså utropstecken och mellanrum).

\section*{Input}
Input består utav en rad text, som avslutas med ny rad (newline).
Texten kan bestå utav vilket tecken som helst i ASCII-intervallet $32$--$126$ (vilket alltså är från och med tecknet mellanrum till och med tecknet tilde), och även tecken nummer $9$ (tab).
Endast tecken listade i tabellen ovan (A--Z, a--z) ska översättas; alla icke-alfabetiska tecken ska skrivas ut (och ej modifieras).
Input kan vara av som mest 10 000 tecken.
\section*{Output}
Outputen ska vara input-texten där varje bokstav (gemener och versaler) översatts till dess motsvarighet i det nya alfabetet.
